O Windows 10 estabelece 4GB como limite para memória física em versões 32-bits e 2TB em versões 64-bits, com exceção da versão Home, que possui limite de 128GB em sua versão de 64-bits~\cite{w10_memory}.
	
	A memória física pode ser dividida em~\cite{w10_memory}:
	\begin{itemize}
		\item \textbf{Reservada para o Hardware}: armazena drivers de hardware que devem sempre permanecer na memória física, não estando disponível para uso do gerenciador de memória.
		
		\item \textbf{Em uso:} É a memória em uso por todos os processos em execução, \emph{kernel} do SO e \emph{drivers}.
		
		\item \textbf{Modificada:} É a memória de páginas que foram modificadas em processos que ficaram em espera. Os dados anteriores são escritos em disco, mas facilmente recuperados.
			
		\item \textbf{Em espera:} É a memória que estava alocada em processos que terminaram normalmente. O gerenciador de memória mantém os dados em memória como uma espécie de cache para arquivos usados recentemente. A memória em espera está disponível para alocação, mas suas páginas são classificadas de 0 a 7, sendo as páginas com menores valores usadas primeiro.
		
		\item \textbf{Livre:} É a memória que ainda não foi alocada ou que retornou para o gerenciador de memória por um processo que foi terminado.
	\end{itemize}

	O gerenciador de memória do Windows 10 faz parte do Windows executive, uma porção em baixo nível do seu \emph{kernel}, residindo no arquivo \emph{Ntoskrnl.exe}. É responsável, entre outras funções, por~\cite{internals_pt2}:
	\begin{itemize}	
		\item Alocar, desalocar e gerenciar a memória virtual, que em sua maior parte está exposta por meio da API do Windows ou de interfaces para drivers de dispositivos em modo kernel;
		\item Garantir que processos não acessem regiões a que não possuem permissão;
	\end{itemize}
	
	O ambiente Windows, de modo geral, utiliza o conceito de espaço de endereçamento virtual para um processo, sendo este o conjunto de endereços da memória virtual que esse processo tem acesso. O espaço de endereçamento é privado e não pode ser acessado por outros processos que não o compartilhem~\cite{internals_pt2}.
	
	O espaço de endereçamento em versões 32-bits do Windows é de até 4GB, dividido em uma partição para o processo e outra para uso do sistema. Versões 64-bits do sistema suportam endereçamento em modo usuário de até 8TB~\cite{w10_memory}.
	
	Assim como todos os componentes do \emph{Windows executive}, o gerenciador de memória é totalmente reentrante, ou seja, pode executado novamente antes que a execução anterior tenha sido concluída, e suporta execução simultânea em sistemas multiprocessados. Isso permite que duas ou mais \emph{threads} adquiram recursos de forma que seus dados não sejam corrompidos~\cite{internals_pt2}.

	Uma funcionalidade importante introduzida no Windows 10 é a compressão de memória. Normalmente, quando uma página está inativa por um longo tempo o gerenciador de memória a move para a área de memória modificada. Se essa página continuar a não ser referenciada mas o sistema necessitar de memória adicional para outros processos a página é então escrita em disco em um arquivo denominado \emph{pagefile}~\cite{w10_memory}.
	
	O Windows 10 introduz o conceito de que, antes da página ser enviada ao disco, o gerenciador de memória irá comprimir páginas não utilizadas. A compressão de memória chega a ocupar apenas 40\% do tamanho original, reduzindo a necessidade de acesso ao disco a 50\% do que era necessário em versões anteriores do Windows~\cite{memory_compression}.