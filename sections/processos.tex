No Windows, processos são detentores de recursos que armazenam informações sobre o espaço de endereçamento virtual, os manipuladores que referenciam os objetos no modo \emph{kernel} e as \emph{threads}. As \emph{threads}, por sua vez, são abstrações do \emph{kernel} para realizar o escalonamento da CPU. Cada uma possui uma prioridade baseada no processo ao qual elas fazem parte e também podem ter afinidades com certos processadores. Cada \emph{thread} possui duas pilhas, uma para que ela seja executada em modo \emph{kernel} e outra para execução em modo usuário~\cite{tanenbaum}.

As \emph{threads} no Windows 10 são híbridas e possuem duas pilhas, uma para que ela seja executada em modo \emph{kernel} e outra para execução em modo usuário.

\textbf{Comunicação entre Processos}: Se dá pela comunicação entre threads, que pode ser feita por meio de \emph{pipes}, \emph{pipes} nomeados, \emph{mailslots}, \emph{sockets}, chamadas de procedimento remotas e arquivos compartilhados~\cite{tanenbaum}.

Um \emph{pipe} é um tipo de pseudo-arquivo que pode ser usado para conectar dois processos. \emph{Mailslots} e \emph{sockets} são similares aos \emph{pipes}. No entanto, diferente dos \emph{pipes}, a comunicação não é bidirecional em \emph{mailslots} e os \emph{sockets} costumam conectar processos em máquinas diferentes. 

\textbf{Sincronização}: O Windows fornece vários mecanismos de sincronização, incluindo semáforos, \emph{mutexes}, eventos e regiões críticas~\cite{tanenbaum}.

\textbf{Escalonamento}: O sistema de escalonamento do Windows é completamente preemptivo e guiado por prioridades.
O \emph{kernel} do Windows não possui uma \emph{thread} de escalonamento central. Por isso, quando uma \emph{thread} não pode mais executar, ela entra em modo \emph{kernel} e executa o escalonador. Para que uma \emph{thread} execute o escalonador, uma das seguintes situações deve ocorrer~\cite{tanenbaum}:
\begin{itemize}
	\item A \emph{thread} atual bloqueia em um semáforo, \emph{mutex} ou evento de E/S;
	\item A \emph{thread} sinaliza um objeto;
	\item O seu quantum acaba.
\end{itemize}

O escalonador também pode ser chamado quando uma operação de E/S ou uma espera temporizada termina.

Para implementar o escalonamento, o sistema mantém uma lista com 32 entradas, tal que cada uma contém um conjunto de \emph{threads} com prioridade correspondente ao número da entrada. O algoritmo de escalonamento busca o vetor desde a prioridade 31 até 0 por uma entrada não vazia. Quando encontra, seleciona a primeira \emph{thread} e esta é executada durante um quantum. 

A API do Windows organiza os processos de acordo com a classe de prioridade que eles recebem quando são criados. Essas classes são: tempo real, alta, acima do normal, normal, abaixo do normal ou ocioso. Feito isso, ela atribui uma prioridade relativa às \emph{threads} individuais dentro de cada processo. Essas prioridades são: tempo crítico, mais alta, acima do normal, mais baixa e ociosa. No \emph{kernel}, a classe de prioridade é convertida para uma prioridade-base e nela aplica-se um diferencial de acordo com a prioridade relativa da \emph{thread}. Desta forma, processos possuem apenas uma prioridade-base, enquanto as \emph{threads} possuem duas prioridades: atual e base. 

As threads de tempo real nunca têm suas prioridades alteradas. As demais \emph{threads}, no entanto, podem ter suas prioridades atuais alteradas. As seguintes situações podem acarretar na alteração da prioridade atual de uma \emph{thread}:
\begin{itemize}
	\item Uma operação de E/S é finalizada;
	\item Uma \emph{thread} que esteja esperando um semáforo, \emph{mutex} ou outro evento seja liberada;
	\item Uma \emph{thread} de \emph{GUI} desperta.
\end{itemize}

Se a \emph{thread} gastar todo o seu quantum durante a execução, sua prioridade é rebaixada. Isso ocorre quantas vezes ela for executada, até que a sua prioridade iguale-se à do nível-base.
